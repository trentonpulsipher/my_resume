\documentclass[11pt,]{article}
\usepackage[sc, osf]{mathpazo}
\usepackage{amssymb,amsmath}
\usepackage{ifxetex,ifluatex}
\usepackage{fixltx2e} % provides \textsubscript
\ifnum 0\ifxetex 1\fi\ifluatex 1\fi=0 % if pdftex
  \usepackage[T1]{fontenc}
  \usepackage[utf8]{inputenc}
\else % if luatex or xelatex
  \ifxetex
    \usepackage{mathspec}
  \else
    \usepackage{fontspec}
  \fi
  \defaultfontfeatures{Ligatures=TeX,Scale=MatchLowercase}
\fi
% use upquote if available, for straight quotes in verbatim environments
\IfFileExists{upquote.sty}{\usepackage{upquote}}{}
% use microtype if available
\IfFileExists{microtype.sty}{%
\usepackage{microtype}
\UseMicrotypeSet[protrusion]{basicmath} % disable protrusion for tt fonts
}{}
\usepackage[margin=1in]{geometry}




\setlength{\emergencystretch}{3em}  % prevent overfull lines
\providecommand{\tightlist}{%
  \setlength{\itemsep}{0pt}\setlength{\parskip}{0pt}}
\setcounter{secnumdepth}{0}
% Redefines (sub)paragraphs to behave more like sections
\ifx\paragraph\undefined\else
\let\oldparagraph\paragraph
\renewcommand{\paragraph}[1]{\oldparagraph{#1}\mbox{}}
\fi
\ifx\subparagraph\undefined\else
\let\oldsubparagraph\subparagraph
\renewcommand{\subparagraph}[1]{\oldsubparagraph{#1}\mbox{}}
\fi

% Now begins the stuff that I added.
% ----------------------------------

% Custom section fonts
\usepackage{sectsty}
\sectionfont{\rmfamily\mdseries\large\bf}
\subsectionfont{\rmfamily\mdseries\normalsize\itshape}


% Make lists without bullets
\renewenvironment{itemize}{
  \begin{list}{}{
    \setlength{\leftmargin}{1.5em}
  }
}{
  \end{list}
}


% Make parskips rather than indent with lists.
\usepackage{parskip}
\usepackage{titlesec}
\titlespacing\section{0pt}{12pt plus 4pt minus 2pt}{4pt plus 2pt minus 2pt}
\titlespacing\subsection{0pt}{12pt plus 4pt minus 2pt}{4pt plus 2pt minus 2pt}

% Use fontawesome. Note: you'll need TeXLive 2015. Update.
\usepackage{fontawesome}

% Fancyhdr, as I tend to do with these personal documents.
\usepackage{fancyhdr,lastpage}
\pagestyle{fancy}
\renewcommand{\headrulewidth}{0.0pt}
\renewcommand{\footrulewidth}{0.0pt}
\lhead{}
\chead{}
\rhead{}
\lfoot{
\cfoot{\scriptsize  Trenton Pulsipher - Trenton Pulsipher }}
\rfoot{\scriptsize \thepage/{\hypersetup{linkcolor=black}\pageref{LastPage}}}

% Always load hyperref last.
\usepackage{hyperref}
\PassOptionsToPackage{usenames,dvipsnames}{color} % color is loaded by hyperref

\hypersetup{unicode=true,
            pdftitle={Trenton Pulsipher:  Trenton Pulsipher (Curriculum Vitae)},
            pdfauthor={Trenton Pulsipher},
            pdfkeywords={R Markdown, resume, template},
            colorlinks=true,
            linkcolor=blue,
            citecolor=Blue,
            urlcolor=blue,
            breaklinks=true, bookmarks=true}
\urlstyle{same}  % don't use monospace font for urls

\begin{document}


\centerline{\huge \bf Trenton Pulsipher}

\vspace{2 mm}

\hrule

\vspace{2 mm}

\moveleft.5\hoffset\centerline{Sr Data Scientist, Henry Schein One}
\moveleft.5\hoffset\centerline{Henry Schein One, 1220 S 630 E, American Fork, UT 84003}
\moveleft.5\hoffset\centerline{ \faEnvelopeO \hspace{1 mm} \href{mailto:}{\tt \href{mailto:tcpulsipher@hotmail.com}{\nolinkurl{tcpulsipher@hotmail.com}}} \hspace{1 mm}  \faPhone \hspace{1 mm}  801 380 6878  \hspace{1 mm}  \faGithub \hspace{1 mm} \href{http://github.com/trentonpulsipher}{\tt trentonpulsipher} \hspace{1 mm}     }

\vspace{2 mm}

\hrule


\subsection{Skills}\label{skills}

Machine Learning \textbar{} Data Mining \textbar{} Visualization
\textbar{} Statistics \textbar{} Big Data \textbar{} R \textbar{}
Tableau \textbar{} SQL

\subsection{Employment}\label{employment}

\subparagraph{Henry Schein One}\label{henry-schein-one}

\emph{Sr Data Scientist} (American Fork, UT) \hfill Apr 2017 - Present

I lead the data science effort at HS One, including being the primary
promoter of data governance, business process and data warehouse
documentation, and analytical best practices.

With the help of interns that I mentor/manage, we built a robust machine
learning predictive modeling and reporting framework to speed delivery
of our data science project deliverables. Using machine learning
algoritms we identified customers at-risk of attrition (saving nearly
\$10M in 3 years) and improved marketing messaging and sales lead
generation through enhanced customer segmentation.

We created web-based tools to share analytical results with the sales,
customer success, and marketing teams internally. Such tools were built
from scratch in Linux to share interactive JavaScript based apps via
HTML pages in a Markdown/Hugo generated website using R as the general
engine.

I also support the business operations team's effort to improve the data
warehouse architecture, dimensional modeling, and reporting services
(Salesforce and Tableau).

\subparagraph{Category Partners}\label{category-partners}

\emph{Sr Analyst} (Idaho Falls, ID) \hfill Apr 2016 - Apr 2017

As the senior analyst of a five person startup I led the analysis,
visualization/dashboard efforts of IRI/Nielsen syndicated data for
grower/shipper clients in the produce category.

\subparagraph{The Church of Jesus Christ of Latter-day
Saints}\label{the-church-of-jesus-christ-of-latter-day-saints}

\emph{Data Analyst / Statistician} (Salt Lake City, UT) \hfill Dec 2014
- Apr 2016

As a data analyst in the Missionary Department I was responsible for all
data science related analytical efforts within the department,
including: data wrangling, modeling, predictive analytics,
visualization, data discovery/exploration, and communication of results
to the highest level of leaders/executives in the Church. I established
knowledge sharing of advanced analytics best practices.

\subparagraph{Pacific Northwest National
Laboratory}\label{pacific-northwest-national-laboratory}

\emph{Statistician} (Richland, WA) \hfill Jul 2009 - Dec 2014

Implemented Bayesian model averaging (BMA) as an ensemble-based machine
learning method to incorporate predictions from multiple experts or
model systems. Successful application of BMA include modeling of
response surfaces numerically generated from multiple expert-derived 3D
conceptual models characterizing sites for carbon sequestration, and
have improved predictive performance for ensemble-based applications in
image classification (computer vision), power grid demand modeling, pKa
protein analysis, and sensor degradation signature discovery. Coauthored
multiple peer-reviewed journal articles and conference papers
demonstrating expertise in BMA related applications. Contributed to the
development of three invention disclosures and one non-provisional
patent submission (pub number: 2013/0297,538).

Developed statistical methodology to spatially allocate weather similar
regions using climate data. Reduction of the number of weather similar
regions and of the number of building types will save climate modelers
and power grid researchers at least hundreds of hours of computation and
analysis time when using PNNL's complex regional earth systems
simulation modeling framework, PRIMA. Formed a two-tiered PAM clustering
approach to adequately describe a subset of representative building
types. Managed Big Data and parallelized statistical computations using
datadr and Rhipe, R packages designed to interact with the Hadoop
framework via Map/Reduce on PNNL's institutional supercomputer and a
Hadoop cluster.

Performed multiple complex data analyses for other project work related
to a variety of areas, including: building energy modeling, power grid
modeling, bio-forensics of biowarfare growth media, river salmon
population studies, air chemical analysis, uncertainty quantification
for visualization of carbon sequestration, and several early drug
development clinical trial related analyses.

\subparagraph{Massachusetts General
Hospital}\label{massachusetts-general-hospital}

\emph{Biostatistician} (Boston, MA) \hfill Jun 2007 - Jul 2009

Tested models for early detection of ovarian cancer in clinical trial
data. Examined longitudinal protein biomarker profiles from blood/serum.
Calculated sensitivity and specificity for predicted data from logistic
regression of potential biomarker combinations from cancer case/control
studies.

\subsection{Education}\label{education}

\emph{Brigham Young University}, MS Statistics \hfill 2007

\emph{Brigham Young University}, BS Statistics \hfill 2005

\subsection{Publications}\label{publications}

\paragraph{Journal Articles}\label{journal-articles}

Hejazi MI, Voisin N, Lui L, Bramer LM, Fortin DC, Hathaway JE, Huang M,
Leung LR, Li HY, Patel PL, \textbf{Pulsipher TC}, Rice JS, Tesfa TK,
Vernon CR, Zhou Y. 2015. ``21st century United States emissions
mitigation could increase water stress more than the climate change it
is mitigating.'' \emph{Proc Natl Acad Sci USA} 112(34):10635-40.

Gosink LJ, Bensema K, \textbf{Pulsipher TC}, Obermaier H, Henry MJ,
Childs H, Joy K. 2013. ``Characterizing and Visualizing Predictive
Uncertainty in Numerical Ensembles Through Bayesian Model Averaging.''
\emph{IEEE Transactions on Visualization and Computer Graphics}
19(12):2703-2712

Gosink LJ, Hogan EA, \textbf{Pulsipher TC}. 2013. ``Bayesian Model
Aggregation for Ensemble-Based Estimates of Protein pKa Values.''
\emph{Proteins: Structure, Function, Bioinformatics} 82(3):354-363

Vlachopoulou M, Gosink LJ, \textbf{Pulsipher TC}, Ferryman T, Zhou N,
Tong J. 2013. ``An Ensemble Approach for Forecasting Net Interchange
Schedule.'' \emph{Power and Energy Society (PES) General Meeting 2013
IEEE}

Stephan CN, Amidan B, Trease H, Guyomarc'h P, \textbf{Pulsipher TC},
Byrd JE. 2013. ``Morphometric Comparison of Clavicle Outlines from 3D
Bone Scans and 2D Chest Radiographs: A Shortlisting Tool to Assist
Radiographic Identification of Human Skeletons.'' \emph{Journal of
Forensic Sciences} 59(2):306-313

Nackos AN, Truong TV, \textbf{Pulsipher TC}, Kimball JA, Tolley HD,
Robison RA, Bartholomew CH, Lee ML. 2011. ``One-step conversion of
dipicolinic acid to its dimethyl ester using monomethyl sulfate salts
for GC-MS detection of bacterial endospores.'' \emph{Anal. Methods}
2011(3):243-258

Yurkovetsky Z, Skates SJ, Lomakin A, Nolen B, \textbf{Pulsipher TC},
Modugno F, Marks J, Godwin A, Gorelik E, Jacobs I, Menon U, Lu K,
Badgwell D, Bast Jr RC, Lokshin AE. 2010. ``Development of a Multimarker
Assay for Early Detection of Ovarian Cancer.'' \emph{Journal of Clinical
Oncology} 28(13):2159-2166

Addona TA, Abbatiello SE, Schilling B, Skates SJ, Mani DR, Bunk DM,
Spiegelman CH, Zimmerman LJ, Ham AL, Keshishian H, Hall SC, Allen S,
Blackman RK, Borchers CH, Buck C, Cardasis HL, Cusack MP, Dodder NG,
Gibson BW, Held JM, Hiltke T, Jackson A, Johansen EB, Kinsinger CR, Li
J, Mesri M, Neubert TA, Niles RK, \textbf{Pulsipher TC}, Ransohoff D,
Rodriguez H, Rudnick PA, Smith D, Tabb DL, Tegeler TJ, Variyath AM,
Vega-Montoto LJ, Wahlander A, Waldemarson S, Wang M, Whiteaker JR, Zhao
L, Anderson NL, Fisher SJ, Liebler DC, Paulovich AG, Regnier FE, Tempst,
Carr SA. 2009. ``Multi-site assessment of the precision and
reproducibility of multiple reaction monitoring-based measurements of
proteins in plasma.'' \emph{Nature Biotechnology} 27:633-641

\paragraph{Other Publications}\label{other-publications}

Baker NA, Dowling C, Gosink L, \textbf{Pulsipher TC}, Sansone SA. 2014.
``Informatics Approaches to Data Preservation and Analysis in Protein
Electrostatics.'' \emph{Biophysical} 108-2(369A)

\subsection{Awards}\label{awards}

\paragraph{R\&D 100 Award}\label{rd-100-award}

2015 R\&D Magazine. ``Power Model Integrator: A system for more accurate
energy forecasts.''

\paragraph{Patent}\label{patent}

``System and Method of Designing Models in a Feedback Loop.'' Patent
issuer and numberus 13/869,290

\end{document}
